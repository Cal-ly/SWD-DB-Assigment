\chapter{WinForm Application}
\label{chapter:winform-application}

\section{Considerations regarding the application}

\subsection{Scope}
Based on the task description and my interpretation of the intention behind the task, the application is the tertiary priority - implying that the database is the primary focus, and diagrams and arguments for documentation are secondary.
Therefore, I have narrowed down the task to an application that can perform CRUD operations on the "Facilities" table in the database. This leads me to the following functionality:

\begin{itemize}
    \item Connection to my local database
    \item Displaying the "Facilities" table
    \item Adding a record to "Facilities" using DB autoincrement
    \item Updating the Name and Description of a record in "Facilities"
    \item Deleting a record from "Facilities"
\end{itemize}

\subsection{Out of scope}
This leads me to the conclusion that I will not implement features such as adding a Facility to a Hotel through HotelFacilities. 
This is because it would require more work on the UI side, rather than on and in the database work.

\subsection{However}
I have added a DataGridView for the "ViewHotelFacilities", which is a view that I have created in the database.
The script can be found in \ref*{label:sql-view-hotel-facilities}. 
This is because I wanted to see how the relationship between the Hotel and the Facilities would be displayed in a DataGridView.

\subsection{Design}
The application will be a simple WinForm application with a DataGridView for displaying the "Facilities" table, and a few buttons for the CRUD operations.
Classic WinForm controls will be used, and the application will be designed to be as simple as possible. 
The reason is the same as the one for the scope of the application: the database is the primary focus, and the application is secondary.


\section{Implementation}
A new WinForm application was created in Visual Studio 2022 using .NET 8.0. The application is named HotelFacilityApp. It is uploaded to GitHub, and can be found at \url{}
