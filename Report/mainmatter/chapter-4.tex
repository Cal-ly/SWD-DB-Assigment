\chapter{WinForm Application}
\label{chapter:winform-application}

\section{Considerations regarding the application}

\subsection{Scope}
Based on the task description and my interpretation of the intention behind the task, the application is the tertiary priority - implying that the database is the primary focus, and diagrams and arguments for documentation are secondary.
Therefore, I have narrowed down the task to an application that can perform CRUD operations on the "Facilities" table in the database. This leads me to the following functionality:

\begin{itemize}
    \item Connection to my local database
    \item Displaying the "Facilities" table
    \item Adding a record to "Facilities" using DB autoincrement
    \item Updating the Name and Description of a record in "Facilities"
    \item Deleting a record from "Facilities"
\end{itemize}

\subsection{Out of scope}
This leads me to the conclusion that I will not implement features such as adding a Facility to a Hotel through HotelFacilities. 
This is because it would require more work on the UI side, rather than on and in the database work.

\subsection{However}
I have added a DataGridView for the "ViewHotelFacilities", which is a view that I have created in the database.
The script can be found in \ref*{label:sql-view-hotel-facilities}. 
This is because I wanted to see how the relationship between the Hotel and the Facilities would be displayed in a DataGridView.

\subsection{Design}
The application will be a simple WinForm application with a DataGridView for displaying the "Facilities" table, and a few buttons for the CRUD operations.
Classic WinForm controls will be used, and the application will be designed to be as simple as possible. 
The reason is the same as the one for the scope of the application: the database is the primary focus, and the application is secondary.

\section{Implementation}
A new WinForm application was created in Visual Studio 2022 using .NET 8.0. The application is named HotelFacilityApp. 
It has been uploaded to GitHub, and can be found at \url{https://github.com/Cal-ly/SWD-DB-Assigment.git} along with the rest of the project.
Be aware, that the application is using a local database, and the connection string in the code (or app.config file) will need to be changed to match your local database.

\subsection{Connection to the database}
The connection to the database is established in the ConnectForm.cs, which is the first form that is shown when the application is started. 
From there the user will connect to the default database, requested to supply their own string or get an error message. 
The connection in the other "slave" forms (CreateForm and UpdateForm) is handled via the \emph{Using} statement, which ensures that the connection is closed when the form is closed.

\subsection{Displaying the "Facilities" table}
The ConnectForm.cs contains a DataGridView that is populated with the "Facilities" table from the database. 
This is simply done using a SQL query:
\mint{sql}|SELECT * FROM Facilities| 
with the C\# side handled with SqlCommand and SqlDataReader.

\subsection{Adding a record to "Facilities" using DB autoincrement}
The CreateForm.cs contains a TextBox for the Name and a RichTextBox for the Description. 
When the user clicks the "Create" button, the data is inserted into the "Facilities" table using a SQL query:
\mint{sql}|INSERT INTO Facilities (Name, Description) VALUES (@FacilityName, @FacilityDescription)|

\subsection{Updating the Name and Description of a record in "Facilities"}
The ConnectForm.cs contains a DataGridView that is populated with the "Facilities" table from the database. 
When the user clicks a rowheader in the DataGridView, the Name and Description of the selected row is displayed in TextBoxes. 
When the user clicks the "Update" button, the data is updated in the "Facilities" table using a SQL query:
\mint{sql}|UPDATE Facilities SET Name = @FacilityName, Description = @FacilityDescription WHERE Id = @FacilityId|

\subsection{Deleting a record from "Facilities"}
The ConnectForm.cs contains a DataGridView that is populated with the "Facilities" table from the database. 
When the user clicks a rowheader in the DataGridView, the Name and Description of the selected row is displayed in TextBoxes.
When the user clicks the "Delete" button, the data is deleted from the "Facilities" table using a SQL query:
\mint{sql}|DELETE FROM Facilities WHERE Id = @FacilityId|

\subsection{Alternativ to Add Facility}
In the UpdateForm.cs, there is also an \emph{Add as New} button. This is a simple way to either add a new or clone it, while stille giving it a new ID.